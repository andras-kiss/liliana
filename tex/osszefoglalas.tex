\chapter{Összefoglalás és következtetések}
\pagestyle{headings}

Diplomamunkám során a pásztázó referencia elektród technikához szükséges mikroméretű referencia elektródokat korábban kidolgozott eljárások alapján terveztem meg és készítettem el. A polarizált grafit céltárgyat vizsgáltam először. Az eredmények ezen a jól jellemzett és sokszor tanulmányozott rendszeren a tapasztalatoknak és várakozásnak megfeleltek. Így az általam készített elektród és kiértékelési módszerek alkalmasnak bizonyultak, hogy azokat korróziós vizsgálatokban használjam. A gyakorlati példaként vas-cink galvánpárt vizsgáltam. A kidolgozott technikával feltérképeztem a működő galvánpár közelében kialakuló elektromos mezőt. A várkozásoknak megfelelően a cink anódként, a vas katódként viselkedett az aktív cellában. A potenciáltérképek mindkét esetben a szakirodalomban leírtakkal megegyeztek, így felhasználhattam az \emph{Eredmények} című fejezetben leírt adatokat a további kiértékeléshez.

A különböző tudományos források ismeretében numerikus deriválással meghatároztam az elektromos mezőt, majd ezt felhasználva a desztillált víz fajlagos vezetőképességének ismeretében az áramsűrűseget is ki tudtam számítani és ábrázolni. Ezen áramsűrűség térképek a galvanikus korrózió során lejátszódó korróziós reakciók sebességéről nyújtanak felvilágosítást, hiszen a meghatározott áramsűrűség és a reakciósebesség közötti összefüggések ismertek.

A technika korábban nem állt rendelkezésre Pécsen. Munkám fő eredménye a pásztázó referencia elektród technikához szükséges elektródok és kiértékelési módszerek kidolgozása a PTE TTK Általános és Fizikai Kémia Tanszékén, és a működés bemutatása néhány egyszerű rendszeren. A \emph{Célkitűzés} című fejezetben pontokban megfogalmazott célokat sikerült mind megvalósítani.
