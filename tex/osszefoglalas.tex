\chapter{Összefoglalás és következtetések}
\pagestyle{headings}

Diplomamunkám során szakirodalmakban leírt technikákat alkalmazva végeztem el a méréseket. A SRET-hez szükséges mikroméretű referencia elektródokat korábban kidolgozott eljárások alapján terveztem meg és készítettem el. A módszer működőképességét a katódosan és anódosan pozitívan polarizált grafit céltárgyon sikeresen bemutattam. A mérések eredményei ezen a jól jellemzett és sokszor tanulmányozott rendszeren a várakozásnak megfeleltek. Így ez a technika alkalmasnak bizonyult, hogy korróziós vizsgálatban mikroméretű referenciaelektródként használjam. A gyakorlati példaként választott vas-cink céltárgyon is elvégezve a pásztázásokat, szintén sikerült bebizonyítani, hogy a céltárgy, a szakirodalomnak megfelelően galvánpárként viselkedett. A cink anódként, a vas katódként viselkedett az aktív cellában. Az \emph{Eredmények} című fejezetben leírt mérési adatok alapján, a valóságot jól tükröző potenciáltérképeket lehet készíteni ezzel a módszerrel. 

A különböző tudományos források ismeretében, kiszámoltam az elektromos mezőt, illetve ezen adatainak és a \emph{Módszerek} című fejezetben leírt desztilállt víz fajlagos vezetőképességének ismeretében az áramsűrűseget is ki tudtam számítani és ábrázolni. Ezen ábrák is a várakozásoknak megfelelőek voltak, az értékek pedig valósághűek.
A \emph{Célkitűzés} című fejezetben pontokban megfogalmazott célokat sikerült mind megvalósítani.

Összegezve, elmondható, hogy a pásztázó referencia elektród technika rendkívül jól alkalmazható korróziós vizsgálatokban, helyettesítve a pásztázó vibráló elektród technikát. Mindkét módszer kitűnően alkalmazható az elektromos mező feltérképezésére.
