\chapter{Összefoglalás és következtetések}
\pagestyle{headings}

Diplomamunkám során a SRET-hez szükséges mikroméretű referencia elektródokat korábban kidolgozott eljárások alapján terveztem meg és készítettem el. A katódosan és anódosan pozitívan polarizált grafit céltárgyat vizsgáltam először. Az eredmények ezen a jól jellemzett és sokszor tanulmányozott rendszeren a tapasztalatoknak és várakozásnak megfeleltek. Így az általam készített elektród alkalmasnak bizonyult, hogy korróziós vizsgálatban mikroméretű referenciaelektródként használjam. A gyakorlati példaként választott vas-cink céltárgyon is elvégezve a pásztázásokat, szintén sikerült bemutatni, hogy a módszer működőképes a galvánpár esetében is. A cink anódként, a vas katódként viselkedett az aktív cellában. A potenciáltérképek mindkét esetben a szakirodalomban leírtakkal megegyeztek, így felhasználhattam az \emph{Eredmények} című fejezetben leírt adatokat a további kiértékeléshez.

A különböző tudományos források ismeretében, kiszámoltam az elektromos mezőt, illetve ezen adatainak és a \emph{Módszerek} című fejezetben leírt desztillált víz fajlagos vezetőképességének ismeretében az áramsűrűseget is ki tudtam számítani és ábrázolni. A kapott ábrák is a várakozásoknak megfelelőek voltak, az értékek pedig valósághűek.
A \emph{Célkitűzés} című fejezetben pontokban megfogalmazott célokat sikerült mind megvalósítani.

Összegezve, elmondható, hogy a pásztázó referencia elektród technika rendkívül jól alkalmazható korróziós vizsgálatokban, helyettesítve a pásztázó vibráló elektród technikát. Mindkét módszer kitűnően alkalmazható az elektromos mező feltérképezésére.
