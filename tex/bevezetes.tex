\chapter{Bevezetés}
\pagestyle{headings}

Szakdolgozatomon a Pécsi Tudományegyetem Természettudományi Karának Általános és Fizikai Kémia Tanszékén kezdtem dolgozni 2020-ban. A tanszék elektrokémiai kutatócsoportjának fő kutatási területe a pásztázó elektrokémiai mikroszkóp alkalmazása korróziós és biológiai tárgyú kérdések megválaszolására, illetve a technika továbbfejlesztése. Önálló feladatom volt annak megvizsgálása, hogy lehetséges-e a technikával az elektromos mező feltérképezése, és ahhoz hasonló információ szerzése, mint ami egy másik, jóval bonyolultabb és körülményesebb módszerrel, a pásztázó vibráló elektród technikával. Utóbbit főleg korróziós problémák vizsgálatára használják. Segítségével könnyen mérhető egy felületen lejátszódó oxidációs és redukciós reakciók sebessége, amire a pásztázó elektrokémiai mikroszkóp nem képes. Léteznek ugyan modellszámításon alapuló ionfluxus becslések, de ezek a modell helyességére támaszkodnak, melyek általában nem pontosan írják le a valóságot. Ha ez lehetséges lenne, annak számos előnye lenne amellett, hogy egy jóval egyszerűbb technikával lehetne ugyanazt az analízist elvégezni.

Dolgozatomban bemutatom néhány kísérlettel, hogy ez lehetséges. Először egy már többszörösen jellemzett, kiszámítható viselkedésű rendszert vizsgáltam, bemutatva, hogy a kapott eredmények összhangban vannak az eddigi tapasztalatokkal. Dolgozatom második felében pedig egy mindennapi életből ismert példa vizsgálatán keresztül mutatom be, mire képes a közreműködésemmel továbbfejlesztett pásztázó elektrokémiai mikroszkóp.
