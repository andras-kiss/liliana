\chapter{Bevezetés}
\pagestyle{headings}

Szakdolgozatomon a Pécsi Tudományegyetem Természettudományi Karának Általános és Fizikai Kémia Tanszékén kezdtem dolgozni 2020-ban. A tanszék elektrokémiai kutatócsoportjának fő kutatási területe a pásztázó elektrokémiai mikroszkóp alkalmazása korróziós és biológiai tárgyú kérdések megválaszolására, illetve a technika továbbfejlesztése. A tanszéken az utóbbi, nagyjából egy évben egy új, a pásztázó elektrokémiai mikroszkóppal rokon mérőműszer, pásztázó referencia elektród mikroszkóp épült. Az alkalmazási terület hasonló, mint a pásztázó elektrokémiai mikroszkópé, azonban ezzel az elektromos mező térképezhető, melyből aztán további származtatott mennyiségek térképei állíthatóak elő. 

Önálló feladatom volt egy galvánpár korróziója során kialakuló elektromos mező feltérképezése ezen módszerrel. Ehhez azonban először el kellett készítenem a megfelelő mérőcsúcsokat, és új mérőprogramokat kellett alkalmaznom. A technika egyik fontos, nem evidens aspektusa a kiértékelési módszer, melyet szintén sikerült kidolgoznom. 

Dolgozatomban bemutatom néhány kísérlettel, hogy a pásztázó referencia elektród módszerrel hogyan lehetséges az elektromos mező feltérképezése és ebből az áramsűrűségtérkép meghatározása. Először egy már többszörösen jellemzett, kiszámítható viselkedésű rendszert vizsgáltam, bemutatva, hogy a kapott eredmények összhangban vannak az eddigi tapasztalatokkal. Dolgozatom második felében pedig egy mindennapi életből ismert példa vizsgálatán keresztül mutatom be, mire képes a tanszéken a közreműködésemmel kiépített pásztázó referencia elektród mikroszkóp.
