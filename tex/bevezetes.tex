\chapter{Bevezetés}
\pagestyle{headings}

Szakdolgozatomon a Pécsi Tudományegyetem Természettudományi Karának Általános és Fizikai Kémia Tanszékén kezdtem dolgozni 2020-ban. A tanszék elektrokémiai kutatócsoportjának fő kutatási területe a pásztázó elektrokémiai mikroszkóp alkalmazása korróziós és biológiai tárgyú kérdések megválaszolására, illetve a technika továbbfejlesztése. Szakirodalmi források szerint Isaacs kifejlesztette a pásztázó referencia elektród technikát, amivel lehetségessé vált az elektromos mező térképezése és ebből az áramsűrűség meghatározása. Azonban elég hamar kifejlesztett egy újabb technikát, ami a pásztázó vibráló elektród technika volt.

A tanszéki PEKM-ot munkám során SRET-ként alkalmaztam. Önálló feladatom volt az elektromos mező feltérképezése ezen módszerrel, és ahhoz hasonló információ szerzése, mint ami a másik, jóval bonyolultabb és körülményesebb módszerrel, a pásztázó vibráló elektród technikával kaphatunk. Utóbbit főleg korróziós problémák vizsgálatára használják. Segítségével könnyen mérhető egy felületen lejátszódó oxidációs és redukciós reakciók sebessége, amire a tanszéken is megtalálható pásztázó elektrokémiai mikroszkóp nem képes. Nagy érzékenysége ellenére számos hátránya van, mint például az oldat keverése az analízis alatt. Léteznek ugyan modellszámításon alapuló ionfluxus becslések, de ezek a modell helyességére támaszkodnak, melyek általában nem pontosan írják le a valóságot. Ha ezzel a módszerrel lehetséges lenne az elektromos mező térképezése, annak számos előnye lenne amellett, hogy egy jóval egyszerűbb technikával lehetne ugyanazt az analízist elvégezni.

Dolgozatomban bemutatom néhány kísérlettel, hogy a SRET módszerrel lehetséges az elektromos mező feltérképezése és ebből az áramsűrűség meghatározása. Először egy már többszörösen jellemzett, kiszámítható viselkedésű rendszert vizsgáltam, bemutatva, hogy a kapott eredmények összhangban vannak az eddigi tapasztalatokkal. Dolgozatom második felében pedig egy mindennapi életből ismert példa vizsgálatán keresztül mutatom be, mire képes a közreműködésemmel továbbfejlesztett pásztázó elektrokémiai mikroszkóp.
