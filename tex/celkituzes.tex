\chapter{Célkitűzés} \label{celkituzes}
\pagestyle{headings}

Munkám fő célja a pásztázó referencia elektród technikához szükséges elektródok és módszerek kidolgozása a PTE TTK Általános és Fizikai Kémia Tanszékén, majd ezek működésének bemutatása néhány példán keresztül. Ennek elérése érdekében pontokba foglalva az alábbi célokat tűztem ki magam elé:

\begin{enumerate}
\item  A pásztázó referencia elektród technikában használatos mikroméretű referenciaelektródok megtervezése és elkészítése.
\item  Polarizált grafit modellcéltárgy feletti 2D pásztázás, annak vizsgálatára, hogy működik-e a módszer, helyes-e a kiértékelés; eredmények összehasonlítása a várakozással.
\item  Gyakorlati példa vizsgálata; cink-vas galvánpár korróziós vizsgálata az előző pontban leírtak szerint; potenciál térképezése.
\item  Az előző két pontban mért potenciáltérképek alapján az elektromos mező számolása numerikus deriválással.
\item  Áramsűrűség meghatározása a fajlagos vezetés és az előzőleg kiszámított elektromos mező adatai alapján.
\end{enumerate}


